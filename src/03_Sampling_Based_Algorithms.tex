\documentclass{beamer}

\mode<presentation>
{
  \usetheme{umbc1}
  \setbeamercovered{transparent}
}

\usepackage[english]{babel}
\usepackage[latin1]{inputenc}
\usepackage[T1]{fontenc}

\usepackage[]{datatool, filecontents}
\DTLsetseparator{=}
\DTLloaddb[noheader, keys={key,value}]{values}{./values.dat}
\newcommand{\var}[1]{\DTLfetch{values}{key}{#1}{value}}

\usepackage{tabularray}

\title[RMP v\var{version}]
{Robot Motion Planning}

\subtitle
{Sampling Based Algorithms}

\author[Narayanan]
{A.~Narayanan\inst{1}}

\institute[Technical University of Munich]
{
  \inst{1}
  Department of Informatics\\
}

\subject{Slides}

\AtBeginSubsection[]
{
  \begin{frame}<beamer>{Outline}
    \tableofcontents[currentsection,currentsubsection]
  \end{frame}
}

\begin{document}
\begin{frame}
    \titlepage
  \end{frame}
  
  \begin{frame}{Outline}
    \tableofcontents
  \end{frame}

  \section{Overview of Sampling Based Approaches}

  \begin{frame}{Difficulty with Classical Approaches}
    \begin{itemize}
      \item Curse of dimensionality
      \item Running time increases exponentially with dimensions of configuration space
      \item Several variants of the path planning problem have proven to be PSPACE-hard
    \end{itemize}
  \end{frame}

  \begin{frame}{Drawbacks of Combinatorial Approaches}
    
  \end{frame}

  \section[PRMs]{Probablilistic Roadmaps}

  \begin{frame}
    \frametitle{Multiple Query Roadmaps}
    A multiple query approach tries to capture the connectivity of the
free space as good as possible, such that multiple, different queries
for paths can be answered very fast. In other words: create a
roadmap that is suitable for as many use cases as p
    
  
  \end{frame}

  \begin{frame}
    \frametitle{PRM - Probablistic Roadmaps}
    Basic steps to construct a PRM:
    \begin{enumerate}
      \item sample vertices and keep vertices that do not lie on an obstacle
      \item find neighbour vertices - k-nearest neighbour or neighbours within a specified radius
      \item connect neighbouring vertices with edges (lines) (and check for collisions on connecting line using e.g. discretized line search)
      \item add vertices and edges until roadmap is dense enough
    \end{enumerate}
  
  \end{frame}

  \begin{frame}
    \frametitle{PRM Visualized}
  
  \end{frame}

  \begin{frame}
    \frametitle{Drawbacks}
    PRMs don’t perform well when there are narrow passages.
  
  \end{frame}

  \begin{frame}
    \frametitle{OBPRM - Obstacle Based PRM}
    Obstacle-based PRMs are constructed by sampling only close to
obstacles. During sampling, the first goal is to find a point that
lies inside an obstacle. Then, another point is sampled at an
arbitrary distance to the first point. Using step-wise approximation, a point sufficiently close to the obstacle border is searched.
  
  \end{frame}


\end{document}