\documentclass{beamer}

\mode<presentation>
{
  \usetheme{umbc1}
  \setbeamercovered{transparent}
}

\usepackage[english]{babel}
\usepackage[latin1]{inputenc}
\usepackage[T1]{fontenc}

\usepackage[]{datatool, filecontents}
\DTLsetseparator{=}
\DTLloaddb[noheader, keys={key,value}]{values}{./values.dat}
\newcommand{\var}[1]{\DTLfetch{values}{key}{#1}{value}}

\usepackage{tabularray}

\title[RMP v\var{version}]
{Robot Motion Planning}

\subtitle
{Sampling Based Algorithms}

\author[Narayanan]
{A.~Narayanan\inst{1}}

\institute[Technical University of Munich]
{
  \inst{1}
  Department of Informatics\\
}

\subject{Slides}

\AtBeginSubsection[]
{
  \begin{frame}<beamer>{Outline}
    \tableofcontents[currentsection,currentsubsection]
  \end{frame}
}

\begin{document}
\begin{frame}
    \titlepage
  \end{frame}
  
  \begin{frame}{Outline}
    \tableofcontents
  \end{frame}

  \section{Overview of Sampling Based Approaches}

  \begin{frame}{Difficulty with Classical Approaches}
    \begin{itemize}
      \item Curse of dimensionality
      \item Running time increases exponentially with dimensions of configuration space
      \item Several variants of the path planning problem have proven to be PSPACE-hard
    \end{itemize}
  \end{frame}

  \begin{frame}{Drawbacks of Combinatorial Approaches}
    
  \end{frame}

  \section[PRMs]{Probablilistic Roadmaps}

  \begin{frame}
    \frametitle{Multiple Query Roadmaps}
    A multiple query approach tries to capture the connectivity of the
free space as good as possible, such that multiple, different queries
for paths can be answered very fast. In other words: create a
roadmap that is suitable for as many use cases as p
    
  \end{frame}

  \begin{frame}
    \frametitle{PRM - Probablistic Roadmaps}
    Basic steps to construct a PRM:
    \begin{enumerate}
      \item sample vertices and keep vertices that do not lie on an obstacle
      \item find neighbour vertices - k-nearest neighbour or neighbours within a specified radius
      \item connect neighbouring vertices with edges (lines) (and check for collisions on connecting line using e.g. discretized line search)
      \item add vertices and edges until roadmap is dense enough
    \end{enumerate}
  
  \end{frame}

  \begin{frame}
    \frametitle{PRM Visualized}
  
  \end{frame}

  \begin{frame}
    \frametitle{Drawbacks}
    PRMs don't perform well when there are narrow passages.
  
  \end{frame}

  \begin{frame}
    \frametitle{How to Improve?}
    \begin{itemize}
      \item Increase number of milestones
      \item Random walk
      \item Path Correction
      \item Sample at obstacle boundaries
    \end{itemize}
    
  
  \end{frame}

  \begin{frame}
    \frametitle{OBPRM - Obstacle Based PRM}
    Obstacle-based PRMs are constructed by sampling only close to obstacles. During sampling, the first goal is to find a point that lies inside an obstacle. Then, another point is sampled at an arbitrary distance to the first point. Using step-wise approximation, a point sufficiently close to the obstacle border is searched.
  
  \end{frame}

  \begin{frame}
    \frametitle{Gaussian Sampler}
    \begin{itemize}
      \item The Gaussian sampler addresses the narrow passage problem by sampling from a Gaussian distribution that is biased near the obstacles.
    \end{itemize}

    \textbf{Steps:}
    \begin{enumerate}
      \item first generate a configuration $q_{1}$ randomly from a uniform distribution.
      \item Then a distance step is chosen according to a normal distribution to generate a configuration $q_{2}$ at random at distance step from $q_{1}$.
      \item Both configurations are discarded if both are in collision or if both are collision-free. A sample is added to the roadmap if it is collision-free and the other sample is in collision.
    \end{enumerate}
  
  \end{frame}

  \begin{frame}
    \frametitle{Sampling Inside Narrow Passages - Bridge Planner}
  
    \begin{itemize}
      \item The bridge planner uses a bridge test to sample configurations inside narrow passages.
    \end{itemize}

    \begin{enumerate}
      \item sample 2 configurations $q'$ and $q"$ from a uniform distribution in $\mathcal{Q}$
      \item add to roadmap if collision free
      \item if both are in collision then add $q_{m}$ halfway between them to the roadmap if it is collision free
    \end{enumerate}
  
  \end{frame}

  \begin{frame}
    \frametitle{Resample (random-bounce walk )}
    % not sure if this should be here, check to make sure
    if there is no path found, using random-bounce walk to generate new milestones, improve the C-space connectivity.
    
    A random-bounce walk from q refers to picking a random direction of motion and moving in this direction until a collision occurs, then a new random direction is chosen. The above steps are repeated until the max. distance is reached, then the current configuration q' and the edge (q, q) are inserted into the roadmap.
  
  \end{frame}

  \begin{frame}
    \frametitle{Single Query Roadmaps}
    Single query planners try to solve a single query as fast as possible, without trying to cover the whole free space
    
  \end{frame}

  \begin{frame}
    \frametitle{Single Query PRM}
  
    PRM itself can also be used as single-query planner. In that case, $q_{init}$ and $q_{goal}$ should be inserted to the roadmap at the beginning. The planner should check periodically if the given query can be solved, that is if $q_{init}$ and $q_{goal}$ belong to the same component of the roadmap. At that point, the construction of the roadmap should be aborted
  
  \end{frame}

  \begin{frame}
    \frametitle{Weighted Randomized Tree Expansion}
    \begin{enumerate}
      \item expand trees from start and goal
      \item pick a node with probability = $1/w(x)$, with $w(x)$ being the amount of neighbors within radius (measurement for exploration around x)
      \item sample $k$ points $(y_{1}, \dots, y_{k})$ around $x$
      \item add $y_{i}$ to the tree if
        \begin{itemize}
        \item $1/w(y_{i}) > 1/w(x)$
        \item $y_{i}$ is collision free
        \item $y_{i}$ can see $x$
        \end{itemize}
      \item if a pair of nodes from start tree and goal tree are close and can see each other, then connect them and terminate
    \end{enumerate}  
  
  \end{frame}

  \begin{frame}
    \frametitle{Rapidly Exploring Random Trees}
    \begin{enumerate}
      \item pick $q_{start}$ as the first node
      \item pick a random target location (every $n^{th}$ iteration, choose $q_{goal}$)
      \item find closest vertex in roadmap
      \item extend this vertex towards target location
      \item repeat steps until $q_{goal}$ is reached
    \end{enumerate}

    For faster execution, the tree can be grown from both $q_{start}$ and $q_{goal}$ (RRT-Connect)
  
  \end{frame}

  \begin{frame}
    \frametitle{How to sample}
  
  \end{frame}

  \begin{frame}
    \frametitle{Path Optimization}
  \end{frame}

  \subsection{Expansiveness}

  \begin{frame}
    \frametitle{$\epsilon, \alpha, \beta$ - Expansiveness}

    \begin{itemize}
      \item A PRM has good \texttt{coverage} if the milestones are distributed in such a way that (almost) any point in the free C-space can be connected to one milestone via a straight line.
      \item A PRM has good \texttt{connectivity} if every milestone is reachable from any other milestone
      \item The coverage and connectivity are characterized by the \textbf{Expansiveness} of the space given by $\epsilon, \alpha, \beta$
    \end{itemize}
  
  \end{frame}

  \begin{frame}
    \frametitle{Definition}
    A free space F is $(\epsilon, \alpha, \beta)$-expansive, if it satisfies these conditions:

    \begin{enumerate}
      \item $\mu(V(p)) \geq \epsilon, \forall p \in F$

      $\Longrightarrow$ each point $p$ must see at least an $\epsilon$ fraction of the free space $F$ $\rightarrow$ $F$ is $\epsilon$-good
      \item $\beta - lookout(S) = \{q \in S \,|\,  \mu(V(q)\setminus S) \geq \beta \cdot \mu(F\setminus S)\}$ 
      $\Longrightarrow$ for any $S \subseteq F$, the $\beta - lookout(S)$ is defined as the subset of points $p$ that can see at least a $\beta$-fraction of the complementary space of $S (\equiv F \setminus S)$
      \item $\alpha$ is then defined as the relative volume of these points $p$ to $S$: $\frac{\mu(\beta - lookout(S))}{\mu(S)}$
    \end{enumerate}

    with $\epsilon, \alpha, \beta \in (0, 1], V(\cdot)$ as the set of visible points of a point and $\mu(\cdot)$ denoting the volume of the set of points
    
  
  \end{frame}


\end{document}