\documentclass[handout]{beamer}

\mode<presentation>
{
  \usetheme{umbc1}
  \setbeamercovered{transparent}
}

\usepackage[english]{babel}
\usepackage[latin1]{inputenc}
\usepackage[T1]{fontenc}
\usepackage{amsmath}

\usepackage[]{datatool, filecontents}
\DTLsetseparator{=}
\DTLloaddb[noheader, keys={key,value}]{values}{./values.dat}
\newcommand{\var}[1]{\DTLfetch{values}{key}{#1}{value}}
\newcommand\Tstrut{\rule{0pt}{2.6ex}}       % "top" strut
\newcommand\Bstrut{\rule[-0.9ex]{0pt}{0pt}} % "bottom" strut
\newcommand{\TBstrut}{\Tstrut\Bstrut} % top&bottom struts

\usepackage{tabularray}

\title[RMP v\var{version}]
{Robot Motion Planning}

\subtitle
{Filtering Algorithms}

\author[Narayanan]
{A.~Narayanan\inst{1}}

\institute[Technical University of Munich]
{
  \inst{1}
  Department of Informatics\\
}

\subject{Slides}
\date{}

\AtBeginSubsection[]
{
  \begin{frame}<beamer>{Outline}
    \tableofcontents[currentsection,currentsubsection]
  \end{frame}
}

\begin{document}
\begin{frame}
    \titlepage
  \end{frame}
  
  \begin{frame}{Outline}
    \tableofcontents
  \end{frame}

  \section[]{Kalman Filter}

  \begin{frame}
    \frametitle{Kalman Filter}
  
    \begin{itemize}
        \item Kalman filtering provides a recursive method of estimating the state of a dynamical system in the presence of noise
        \item it simultaneously maintains estimates of both the state vector ($\hat{x}$) and the estimate error covariance matrix ($P$)
        \item Kalman filter is a specific example of a more general technique known as \textbf{probabilistic estimation}
        \item the motion model is assumed to be a linear function of the state variables and the inputs
        \item Errors in both the motion model and the sensor model are assumed to be zero-mean white Gaussian noise
    \end{itemize}
  
  \end{frame}

  \section[]{Extended Kalman Filter}

  \section[]{Unscented Kalman Filter}

  \section[]{Bayes Filter}

  \section[]{Particle Filter}
\end{document}